%%%%%%%%%%%%%%%%%%%%%%%%%%%%%%%%%%%%%%%%%%%%%%%%%%%%%%%%%%%%%%%%%%%
% Lebenslauf.tex
% hier befindet sich der Inhalt zum Lebenslauf
%
% Autor: Gottfried Schrittwieser
% Lizenzhinweis: Siehe License.txt im Wurzelverzeichnis
%%%%%%%%%%%%%%%%%%%%%%%%%%%%%%%%%%%%%%%%%%%%%%%%%%%%%%%%%%%%%%%%%%%


% Die grundlegenden Einstellungen für den Lebenslauf befinden sich in der Datei Einstellungen
% Am Ende eines mehrzeiligen Eintrages, bzw am Ende eines Itemize sollte das Kommando \strut stehen, damit der Abstand nach dem Eintrag richtig berechnet wird.

% Angabe der Längsten Stadt
\laengsteStadt{a.d. LangemFluss}

% Der Titel des Lebenslaufes
\titel[\emph{\nameCV}]

\section{Persönliche Daten}
\eintragK{Name}				{\nachname}
\eintragK{Vorname}			{\vorname}
\eintragK{Akad. Grad}		{\akadTitel}
\eintragK{Geburtsdatum}		{\geburtstag\ in \geburtsort}
%\eintragK{Geburtsort}		{\geburtsort}
\eintragK{Nationalität}		{\nationalitaet}
\eintragK{Anschrift}		{\strasse\newline\PLZwohnort}
%\eintragK{}				{\PLZwohnort}
\eintragK{Telefon}			{\mobilNr}
\eintragK{Email}			{\myEmail}
\eintragKK{Sonstiges}		{Sonstiges 1}{Sonstiges 2}


\section{Berufliche Tätigkeiten}
\eintragL{\zeit[01.2008-09.2012]}{Techniker}{}{Stadt}{%	
	Aufgaben: ...\neueZeile	
	Bei Firma ...\strut
}

\eintragL{\seit[10.2012]}{Regelungstechniker}{}{Stadt}{%	
	Bei Firma ...\strut
}



\section{Universitäre Ausbildung}
\subsection{an der Universität ...}
% Speichere Wert von zweiSpaltig und ändere diesen auf true
\edef\statusZweiSpalten{\gettoggle{zweiSpalten}}
\toggletrue{zweiSpalten}
\eintragL{\zeit[9.2005-8.2009]}{Bachelorstudium ...}{B.Sc.}{}{%	
	\enclRef{ZeugnisBsp}{Abschlussnote: Link zum Zeugnis} \neueZeile
	Bachelorarbeit:\strut
	\begin{itemizeCV}
		\item \glqq ...\grqq 
		\item Themenbereich: ...
		\item verwendete Technologien: ...\strut
	\end{itemizeCV}
}


% Lokales deaktivieren des "Itemize" 
% Aktuellen Status speichern
\edef\itemizeStatus{\gettoggle{itemizeEintrag}}
\togglefalse{itemizeEintrag}
% Eintrag
\eintragL{\zeit[9.2009-11.2012]}{Masterstudium ...}{M.Sc.}{}{
Hier steht ein langer Text\neueZeile
mit mehreren Zeilenumbrüchen, einer Zeile die länger als die Breite der Spalte für zweispaltige Einträge ist und\neueZeile
ohne Aufzählungspunkte.\strut
}
% Alten Status wieder herstellen
\settoggle{itemizeEintrag}{\itemizeStatus}
\settoggle{zweiSpalten}{\statusZweiSpalten}


\section{Schulbildung}
\eintragLZeitStadt{\zeit[09.1997-08.2001]}{\zeit[09.2001-08.2005]}{Gymnasium}{Matura}{Stadt}{a.d. LangemFluss}{
Hier steht ein Text der länger als die Zeilenbreite ist und somit ein Umbruch erzwungen wird.\neueZeile
Wenn in den Einstellungen die Aufzählungspunkte aktiviert sind, dann befinden sich hier zwei Aufzählungspunkte.\strut
}



\section{Weitere Tätigkeiten}
\eintragLEinzeiler{\seit[01.2000]}{Freiwillige Feuerwehr}{}{Stadt}{}
\eintragLEinzeiler{\Datum[10.2010]}{Organisator Feuerwehrfest}{}{Stadt}{}



% Spaltenbreite der linken Spalte im EintragKB
\laengsterLinkerEintragKB{Webtechnologien}

% eintragKB links ausrichten
%\toggletrue{eintragKBlinks}

% Wenn eintragKBlinks ausgewählt ist, dann ist es eventuell auch sinnvoll die Section Breite anzupassen
\iftoggle{eintragKBlinks}{\setBreiteLinkeSpalteLL{\breiteLinkerEintragKB}}{}

\section{IT--Kenntnisse}

\eintragKB{Betriebssysteme}{Windows, Linux}
\eintragKB{Office}{OpenOffice, \LaTeX, und noch viel mehr damit ein Zeilenumbruch erzeugt wird}
\eintragKB{Webtechnologien}{HTML, PHP}


\section{Sonstiges}
\eintragKB{Sprachen}{Englisch: C1\newline Deutsch: Muttersprache}
\eintragKB{Führerschein}{A und B}



\vspace{3\baselineskip}
\signaturCV[, am ]

